\documentclass[a4paper,12pt]{article}
\usepackage[utf8]{inputenc}
\usepackage{amsmath,amssymb,amsthm}
\usepackage{natbib}
\usepackage[margin=3cm]{geometry}
\usepackage{todonotes}
\usepackage{mathtools}
\usepackage{subfiles}
\usepackage{subcaption}
\usepackage{booktabs}

\begin{document}
\title{}
\author{}
\section{Introduction}
\section{Methods}
\subsection{Search Strategy}
\subsection{Simulation Study}
Can adapt from ESA
\section{Results}
\subsection{Scoping Review}
Review descriptions (papers found, filtering, reading etc.)
BOX 1: why important and how were they being used. DESCRIPTION
Datasets are outdated - old!
BOX 2: Data sources and methods. Spatial scale, Temporal scale, (questionnaires, CDR), bias, strengths / limitations
Figure 2 description: (Spatial interaction model description (gravity etc.)
Box 3: What is being modelled: flows, absolute numbers, relative flows, probability of moving, etc. 
Demographic and geographic exploration. Describe why distance, population size distribution etc. are important and how they can be grouped.
Implications for epidemic spread using simulations
PStay DHS data (would be very very valuable)
\subsection{Simulation}
Compare with truth using ``Raw data''
\subsubsection{Wrong country, wrong pstay}
\subsubsection{Aggregated data correct country averaged (at the higher spatial resolution) pstay}
\section{Figures}
Figure 1: (a) Empirical data, (b) Estimated data with details inset.
Figure 2. Schematic 
Figure 3. (b) Time to first case (c) Time to peak (d) Invasion order.
(a) Predicted vs observed; Seeding locations.
Figure 4. Aggregated data used for predicted at
higher resolution (b) Time to first case (c) Time to peak (d) Invasion order.
(a) Predicted vs observed; Seeding locations.


\end{document}